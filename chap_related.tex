\chapter{Related Work} 
\label{sec:related}

As FCC adopt the white space bands for communication, the application
of these new frequency resource become new research spot. Many previous
works are related to the challenges for white space frequencies, but
no one focused on and answered on them.

\section{White Space Links}
\label{subsec:related_whitelinks}

A bunch of work has been done on radio-scene analysis and channel 
identification for utility of channel adaptation dating back to 
Simon Haykin~\cite{haykin2005cognitive}. Some work of Multi-bands/Multi-channels 
in cognitive radios focus on optimize performance, such as avoiding 
frequency diversity~\cite{rahul2009frequency}. In~\cite{OAR} an opportunistic 
algorithm is introduced to balance the cost of spectrum sensing and channel
switching in multichannel scenario.

Oppose to previous works, our work is motivated by prospective releasing 
frequency, white space bands used for TV and exploit the comparison 
across all the available bands in the future. Previous work assume 
all the channels have the same performance in an clean environment 
for channels with small gap of frequencies. However, when the white space
bands are activated, the assumption need to be adjust due to the propagation
variation of the channels with large gap in frequency. 
Some other works focus on the theoretic stopping rules of spectrum 
sensing with the cost of time and channel switching gains
~\cite{sabharwal2007opportunistic, OAR}. In contrast, we focus on
measurement driven framework to exploit the channel selection
rules under in-field environment and to leverage the impacts of 
white space bands on wireless communication links.

\section{White Space Networks}
\label{sec:related_subwhitemesh}

Prior work in wireless network deployment has focused extensively on 
solving gateway placement, channel assignment, and routing problems 
to reduce the interference generated inside the network
~\cite{he2008optimizing,ramachandran2006interference,akyildiz2006next}.
Unfortunately, few works in network deployment notice the interference 
across networks. Some cognitive radio works discuss this inter-network 
interference, but most of them focus on point-to-point communication 
other than taking a network-wide view~\cite{cabric2004implementation}.
 
% White Space large propagation range
With new FCC regulations on the use of white space bands, there are 
two factors to consider with such bands: large propagation range 
and existing inter-network interference from TV stations and other 
devices such as microphones~\cite{fccwhitespace,cui2013leveraging,bahl2009white}.
Prior work does not specifically study the benefits of jointly 
using white space and WiFi bands in the deployment of wireless access 
networks~\cite{akyildiz2005wireless}. Additionally, prior work related 
to white spaces target opportunistic media access. However, the 
application of white spaces across diverse population densities 
has not been fully explored.

% Network propagation difference
Finally, some works discuss the propagation variation in both WiFi 
bands and white space bands. For example, Robinson et al. models 
the propagation variation at the same band in terrain domain
~\cite{robinson2010deploying}. Another work proposes a database-driven 
framework for designing a white space network with database of 
primary user (TV station) locations and channel occupation
~\cite{murty2012senseless}. However, these works do not jointly study 
the influence of white space and WiFi bands on network deployment 
according to their resulting propagation variation and spectrum utilization.

% New Problem: more radios on access points or more access points
% fixme WITH MOA



% Deployment problem 
Wireless mesh network deployment is to design wireless architecture
for offering Internet service in an target area.
There are significant challenges in wireless mesh network deployment,
such as user priorities, user behaviors, long term throughput estimation, selfish clients,
interference and energy efficiency, etc.~\cite{tragos2013spectrum}
These challenges are distinguished under the topics of channel assignment,
cognitive radio, protocol design, etc.~\cite{tragos2013spectrum,akyildiz2006next}
Previous works have recognize the impact of interference in 
wireless mesh network deployment is the key issue~\cite{tang2005interference,irwin2013resource,chieochan2013channel}.
To overcome the challenges, a lot of works have been done to optimize the 
deployment in increasing throughput, minimize resource, reducing interference,
etc.~\cite{irwin2013resource,subramanian2008minimum,doraghinejad2014channel}
Many works have studied the network deployment problem in multihop wireless
networks~\cite{jain2005impact,akyildiz2005wireless,raniwala2004centralized,tragos2013spectrum}.
In addition, multiple radios have been used to improve the routing in multi-channel
scenarios~\cite{draves2004routing,irwin2013resource}. 
Both static and dynamic network deployments have been discussed in previous works under
the 802.11 WiFi scenario~\cite{wu2006analysis,ramachandran2006interference,subramanian2008minimum}. 
However, all of the aforementioned works have not considered propagation 
differences of the diverse frequency bands of white space and WiFi, which we show are 
critical improving the performance of mesh networks.
Frequency agility in multiband scenario brings more traffic capacity to 
wireless network deployment as well as more complexity of resolving the interference issues.

% Network deployment focuses, metrics/targets
Previous work~\cite{pcuiwinmee} involve the inter-network interference in
multiband scenario, but did not offer the solution of intra-network interference.
As a new designed wireless network, intra-network interference is 
more important for performance estimation. Previous work focus on
WiFi wireless networks proposed several methods to reduce the 
interference targeting on multiple metrics.
~\cite{tang2005interference,he2008optimizing,robinson2010deploying}
focus on reducing the gateway mesh nodes. ~\cite{irwin2013resource,subramanian2008minimum} try to reduce the
overall interference in the worst case of traffic independent scenario.
~\cite{chieochan2013channel,li2012channel} improve the performance
in throughput. However, these works fails to involve the traffic demands
of clients in their solutions.~\cite{robinson2010deploying,long2013fair} consider
the QoS requirements in the WiFi network design. Our work also
consider the traffic demands from the client as part of our 
network design to satisfy both customers and vendors.

% Solution methods
The wireless network deployment problem has been 
proved as a NP-hard problem~\cite{doraghinejad2014channel}. 
Several works introduce relaxed linear program formulation 
to find the optimization of multihop wireless networks 
\cite{tang2005interference,irwin2013resource,filippini2013new}.  
Also, game theory methods is another option to solve 
the problem~\cite{raniwala2005architecture,
wang2010game}.  
Social network analysis is also popular in wireless 
network design~\cite{zhu2013survey}.
In this work, we model the problem to a linear program
to approach the optimal solution and generate heuristic
algorithms to find a practical solution for the problem.

% White Space
To be used effectively, white space bands must ensure that available TV bands
exist but no interference exists between microphones and other devices~\cite{bahl2009white}. 
White space bands availability has to be known in prior of network deployment.
TV channels freed by FCC are fairly static in their channel assignment, 
databases have been used to account for white space channel availability 
(e.g., Microsoft's White Space Database~\cite{msdatabase}).
In fact, Google has even visualized the licensed white space channels 
in US cities with an API for research and commercial use~\cite{googledatabase}.
In contrast, we study the performance of mesh networks with a varying number 
of available white space channels at varying population densities, assuming 
such white space databases and mechanisms are in place. As FCC release these 
bands for research, many methods have been proposed to employ these frequency bands.
~\cite{bahl2009white} introduce WiFi like white space link implementation on USRP and 
link protocols. ~\cite{cui2013leveraging} discuss the point to point communication
in multiband scenario. In~\cite{filippini2013new}, white space band application is 
discussed in cognitive radio network for reducing maintenance cost. 
In~\cite{deb2009dynamic}, the white space is proposed to increase the 
data rates through spectrum allocation. 
In our work, we focus on maximizing the throughput of the wireless network.

% Here, change them

% White Space
Since white space bands were free for wireless communication, many efforts have been 
put in the area for the application of white space bands.~\cite{fccwhitespace} 
In~\cite{bahl2009white}. the author considered a cognitive method to avoid collision 
between white space communication and TV broadcasting. 
Many works increasing the convenience of using white space databases have been published 
(e.g., Microsoft's White Space Database~\cite{msdatabase}).
Google has even visualized the licensed white space channels 
in US cities with an API for research and commercial use~\cite{googledatabase}.
Previous work discussed the point to point communication with white space bands~\cite{cui2013leveraging}, 
and the wireless network deployment with plenty white space channels~\cite{pcuiwinmee}.
However, many of the major cities in the US do not have plenty white space channels, such as 
most area of Austin, TX has only one white space channel. As far as we know, there is no work 
discuss these scenarios.


%  Multi-channel
Applying white space in wireless network is similar to the previous multi-channel works other the 
propagation variation. In~\cite{bodas2012low} a multi-channel system is formulated as a queuing 
system and Server Side Greedy algorithm is proposed to optimize the throughput with low complexity. 
In~\cite{ji2013performance}, Delay-based Queue-Side-Greedy algorithm is proposed with low complexity 
for optimal throughput and near-optimal delay.~\cite{liu2014energy} develop a multi-objective optimization 
framework to minimal energy consumption in a multi-channel multi-radio system. 
However, these works do not address minimizing the resource for certain quality of service and assume an 
on-off channel model.

% Need to modify
Previous work studied the multi user setting with a single channel~\cite{tan2010distributed}. 
Spectral diversity is isolated for a single user in~\cite{shu2009throughput}. In~\cite{liu2013stay}, 
multi-user dynamic channel access is proposed jointly consider the temporal and spectral diversity in a multichannel model. However, 
none of these works address the channel association problem in multiband scenario.
Previous work~\cite{pcuiwinmee} studied the 
white space application in access network deployment with spectral diversity. However, these works 
fails to leverage the white space frequency in multi-user diversity in both spectral and temporal scenarios.

%% Multithread
% On-off channel
Previous works in real time systems put many efforts to minimize the resources, such as processors~\cite{nelissen2012techniques}.
In~\cite{li2014analysis}, the author proves the capacity augmentation bounds for schedulers of parallel tasks. 
However, these works assume the parallel tasks have uniform servers. 
Previous work~\cite{chen2011feasibility} investigate the white space in a queuing system without considering the 
heterogeneous topology.
In contrast, we study the performance of a heterogeneous network with both white space channels 
and WiFi channels in channel utilization. 

% gan2014multiple

% weighted sum/less info

