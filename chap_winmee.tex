\chapter{Spectrum Adaptation for Single Hop Access Networks} 
\label{ch:winmee}

% Introduce the content of this section
In this chapter, we illustrate the challenges of band selection
in wireless network deployments and formulate the problem of band 
selection in wireless network deployments jointly using WiFi and white space bands. 
Further, we present a measurement-driven framework for estimating the 
access point number to serve the traffic demand of a given population.


\input{multiband_winmee/problemformulation}
\section{Data Process}
\label{sec:experiment}

Fixme


%Detail may not useful
Parsing script
In our experiments, we are constantly collecting large amounts of data, including the received signal level, current location, velocity, and time of day. Working with the memory-limited Gateworks boards, it became necessary to implement a solution to collect large amounts of data without exhausting the available memory space on the boards. Thus, we compiled a script to parse an undetermined number of data files containing the raw data collected from the ongoing experiments. Utilizing the Perl programming language, which the Gateworks boards are capable of running, the script scans every file in a directory we specify and parses them, looking specifically for signal level, lat/long location, velocity, and time data recorded from the experiments. Upon finding this information, the script reformats the data by placing it into a .csv file. Additionally, using the location data, the script calculates the  distance between the transmitting and receiving boards and adds this information to the .csv file. Upon parsing the data, the raw experimental data files can simply be delete from memory, freeing up space for the data of subsequent experiments.

Activity monitor script
For in-situ experiments, the need became apparent to track the number of new incoming packets and compare it with the number of previously received packets. Additionally, this needed to be done for each of the four wireless radios on the board. In doing this, we identify the most efficient frequency band to transmit data. To implement this system, a script was needed to run efficiently in the background while experiments were taking place. To achieve this, we wrote a bash shell script to run directly on the board without relying on any higher level programming language that could potentially cause greater performance overhead. As a result, the script only consumes one to two percent of CPU resource. The script begins by examining the received bytes across each radio for a length of 30 seconds and placing the bytes received each second on a new line in a file. Upon the completion of the 30 second buffering time period, the next second of received bytes on each radio is read and compared with the last 30 seconds of received data. This ratio of the most recent received data to old received data is then calculated and written to four files, one for each radio, for its subsequent use in selecting which radio to transmit/receive from.


\input{multiband_moa/problemformulation}
\section{Data Process}
\label{sec:experiment}

Fixme


%Detail may not useful
Parsing script
In our experiments, we are constantly collecting large amounts of data, including the received signal level, current location, velocity, and time of day. Working with the memory-limited Gateworks boards, it became necessary to implement a solution to collect large amounts of data without exhausting the available memory space on the boards. Thus, we compiled a script to parse an undetermined number of data files containing the raw data collected from the ongoing experiments. Utilizing the Perl programming language, which the Gateworks boards are capable of running, the script scans every file in a directory we specify and parses them, looking specifically for signal level, lat/long location, velocity, and time data recorded from the experiments. Upon finding this information, the script reformats the data by placing it into a .csv file. Additionally, using the location data, the script calculates the  distance between the transmitting and receiving boards and adds this information to the .csv file. Upon parsing the data, the raw experimental data files can simply be delete from memory, freeing up space for the data of subsequent experiments.

Activity monitor script
For in-situ experiments, the need became apparent to track the number of new incoming packets and compare it with the number of previously received packets. Additionally, this needed to be done for each of the four wireless radios on the board. In doing this, we identify the most efficient frequency band to transmit data. To implement this system, a script was needed to run efficiently in the background while experiments were taking place. To achieve this, we wrote a bash shell script to run directly on the board without relying on any higher level programming language that could potentially cause greater performance overhead. As a result, the script only consumes one to two percent of CPU resource. The script begins by examining the received bytes across each radio for a length of 30 seconds and placing the bytes received each second on a new line in a file. Upon the completion of the 30 second buffering time period, the next second of received bytes on each radio is read and compared with the last 30 seconds of received data. This ratio of the most recent received data to old received data is then calculated and written to four files, one for each radio, for its subsequent use in selecting which radio to transmit/receive from.




\input{multiband_whitecell/problemformulation}
\input{multiband_whitecell/problemstatement}
\section{Data Process}
\label{sec:experiment}

Fixme


%Detail may not useful
Parsing script
In our experiments, we are constantly collecting large amounts of data, including the received signal level, current location, velocity, and time of day. Working with the memory-limited Gateworks boards, it became necessary to implement a solution to collect large amounts of data without exhausting the available memory space on the boards. Thus, we compiled a script to parse an undetermined number of data files containing the raw data collected from the ongoing experiments. Utilizing the Perl programming language, which the Gateworks boards are capable of running, the script scans every file in a directory we specify and parses them, looking specifically for signal level, lat/long location, velocity, and time data recorded from the experiments. Upon finding this information, the script reformats the data by placing it into a .csv file. Additionally, using the location data, the script calculates the  distance between the transmitting and receiving boards and adds this information to the .csv file. Upon parsing the data, the raw experimental data files can simply be delete from memory, freeing up space for the data of subsequent experiments.

Activity monitor script
For in-situ experiments, the need became apparent to track the number of new incoming packets and compare it with the number of previously received packets. Additionally, this needed to be done for each of the four wireless radios on the board. In doing this, we identify the most efficient frequency band to transmit data. To implement this system, a script was needed to run efficiently in the background while experiments were taking place. To achieve this, we wrote a bash shell script to run directly on the board without relying on any higher level programming language that could potentially cause greater performance overhead. As a result, the script only consumes one to two percent of CPU resource. The script begins by examining the received bytes across each radio for a length of 30 seconds and placing the bytes received each second on a new line in a file. Upon the completion of the 30 second buffering time period, the next second of received bytes on each radio is read and compared with the last 30 seconds of received data. This ratio of the most recent received data to old received data is then calculated and written to four files, one for each radio, for its subsequent use in selecting which radio to transmit/receive from.




\section{Summary}
\label{sec:winmeeconclusion}

In this chapter, we jointly considered the use of WiFi and white space bands for 
deploying wireless access networks across a broad range of population densities.
To consider network deployment costs, we proposed a Multiband Access Point Estimation 
framework to find the number of access points required in a given region.
We then performed spectrum utilization measurements in the DFW metropolitan 
and surrounding areas to drive our framework and find the influence of white spaces on
network costs in these representative areas. Through 
extensive analysis across varying population density and channel combinations across bands, 
we show that white space bands can reduce the number of access points by 1650\%
and 660\% in rural and sparse urban areas, respectively. However, the same cost savings
are not achieved in dense urban and downtown type area. Finally, we investigate different 
band combinations in two population densities to show that greater access to white space 
channels have greater total savings of mesh nodes when the total number of channels used 
in the network is fixed (i.e., given a total number of allowable WiFi and white space channels). 
As the population and spectrum utilization increase, the cost savings of white space bands
diminish to the point that WiFi-only channel combinations can be optimal.
Through extensive analysis across varying population density and channel combinations 
across bands, we show that white space bands application in hetergeneous access point 
can reduce the number of access points by 260\% in sufficient white space configuration
and 80\% gain in limited white channel scenarios.
% Whitecell
To consider the power consumption of a system, we 
formulate the heterogeneous wireless network as a queuing system. 
We then analyze the heterogeneous queuing system based on previous queuing 
theory work. To address the resource allocation of white space channels, 
we proposed a Greedy Server-side Replace algorithm to find the resource allocation 
with minimum power consumption. We then perform spectrum utilization 
measurements in the typical areas of downtown, urban, campus, neighborhoods 
to drive the algorithm. We also leverage the user mobility pattern from WiEye 
measurements. Through extensive analysis across the spectrum utilization 
and user mobility, we show that white space bands can reduce the average of 
power consumption by 64.70\% on average over 24 hours. We also integrated previous 
measurements work in north Texas and find the power consumption is reduced by white space
bands by 512.55\% in sparse area.
In the future, we will consider to propose the heterogeneous wireless network 
deployment with large scale user mobility patterns.
% Future work



