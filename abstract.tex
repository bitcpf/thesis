% Challenge in wireless communication
In 2009, the FCC approved the use of broadband services in the white space 
frequency of UHF TV bands, which were formerly exclusively 
licensed to television broadcasters. Here, bands represent the continuous
frequency have same or similar propagation characteristics, for instance 2.4 GHz
band, 5 GHz band; channel notes
a piece of frequency directly used for communication with a small bandwidth, such
as 20 MHz.
These white space bands are now available 
for unlicensed public use, offering new opportunities for the design of devices 
and networks with better performance in terms of throughput and cost. 
People are working to implement the new opportunities in the field to serve people better. 
In this thesis, we seek to exploit these new spectrum opportunities in wireless links
and networks. In particular, while many 
metropolitan areas sought to deploy city-wide WiFi networks, the densest urban 
areas were not able to broadly leverage the technology for large-scale Internet 
access.  Ultimately, the small spatial separation required for effective 802.11 
links in these areas resulted in prohibitively large up-front costs. The far 
greater range of these lower white space carrier frequencies are especially 
critical in rural areas, where high levels of aggregation could dramatically 
lower the cost of deployment and is in direct contrast to dense urban areas in 
which networks are built to maximize spatial reuse. Though both academic and 
industry experts are eagerly looking forward to the application of these white space 
bands, more solid theoretic and practical work need to be done of this topic. 
Here to enable efficient real-world deployments, we investigate spectrum agility in both link communication 
and network deployment to answer part of the questions. 

% List previous papers
% VNC
First, wireless frequency bands is the foundation and one of the key parts 
of further wireless applications. We build a measurement driven algorithm 
to discover the best channel in a vehicular multiband environment. We leverage 
knowledge of in-situ operation across frequency bands with real-time measurements 
of the activity level to select the the band with the highest throughput. To 
do so, we perform a number of experiments in typical vehicular topologies. 
With two models based on machine learning algorithms and an in-situ training 
set, we predict the throughput based on: ({\it i.}) prior performance for 
similar context information ({\it e.g.}, SNR, GPS, relative speed, and link 
distance), and ({\it ii.}) real-time activity level and relative channel quality 
per band. In the field, we show that training on a repeatable route with these 
machine learning techniques can yield vast performance improvements from prior 
schemes. 
% Winmee
Second, leveraging a broad range of spectrum across diverse population densities 
becomes a critical issue for the deployment of data networks with WiFi and white 
space bands. To address this issue we propose a measurement-driven band selection 
framework, Multiband Access Point Estimation (MAPE). Then we apply the framework 
with the data measured the spectrum utility in the Dallas-Fort Worth metropolitan 
and surrounding areas to show the white space band benefit in network deployment. 
In particular, we study the white space and WiFi bands with in-field spectrum 
utility measurements, revealing the number of access points required for an area 
with channels in multiple bands. In doing so, we find that networks with white space 
bands reduce the number of access points by up to 1650\% in sparse rural areas over 
similar WiFi-only solutions. In more populated rural areas and sparse urban areas, 
we find an access point reduction of 660\% and 412\%, respectively.  However, due 
to the heavy use of white space bands in dense urban areas, the cost reductions 
invert (an increase in required access points of 6\%).  Finally, we numerically 
analyze band combinations in typical rural and urban areas and show the critical 
factor that leads to cost reduction: considering the same total number of channels, 
as more channels are available in the white space bands, less access points are 
required for a given area.

% Whiteteris
Furthermore, we compare the heterogeneous access points with simultaneous access to 
multiple frequency bands in a variety configurations to discover the best set of 
spectrum combinations for an access point. In particular, we address a heterogeneous white space and 
WiFi access tier deployment problem and propose a relaxed ILP to get the lower 
bound of the amount of access point under resource limitations and a heuristic 
approach to the problem. Then, we map the problem as a bin packing problem and 
propose to Multiband Heterogeneous Access Point Deployment~(MHAPD) method to address
the problem. In doing so, we discuss the benefit of white space bands 
in reducing the number of access points and provide a heuristic solution for 
wireless network deployment. The numeric results shows that MHAPD gains 260\% 
against the linear hexagon deployment method.  We further analyze the performance 
of heterogeneous access point performance in a variety of population densities. 
The numeric results show that heterogeneous multiband access points could improve the budget 
saving upto 323\%. 

% New
% Whitecell
Moreover, the white spaces resource distribution is restricted by FCC contrast to the population density, 
dense area has few white space channels.
Thus, leveraging the range of spectrum across user mobility becomes a critical issue for the operating of 
heterogeneous data networks with WiFi and limited white space bands. 
% Work
We present a feasibility analysis for a heterogeneous network resource allocation to reduce 
the power consumption via queuing theory approach.
In particular, we study the spectrum utility across multi-bands and the user mobility across a weekday in 
typical environment of the Dallas-Fort Worth metroplex through measurements. 
Moreover, we propose a Greedy Server-side Replace (GSR) algorithm to reduce the power consumption with 
white space channels application. 
In doing so, we find that networks with white space bands reduce the power consumption by up to 512.55\% in sparse 
rural area over WiFi-only solutions via measurements driven numerical simulation. In more populated areas. we find 
an power consumption reduction on average across 24 hours by 24.57\%, 46.27\%,67.40\% over WiFi only network with 
one to three white space channels respectively.
We further investigate the quality of service requirements impacts on power consumption across the number of channels.
We find that the power consumption reduction is up to 150.89\% with three white space channels in dense areas with 
relaxed required waiting time.



% WM
Then, to quantify the spectrum utilization and far greater range of lower carrier 
frequencies on multihop access networks, 
%we consider how these white 
%space bands can be leveraged in large-scale wireless mesh network deployments 
%with in-field measured channel utilization. In particular, 
we present an integer 
linear programming model to leverage diverse propagation characteristics and 
the  channel occupancy of white space and WiFi bands to deploy optimal wireless multiband 
networks. Since such problem is known to be NP-hard, we design a measurement driven 
heuristic algorithm, Band-based Path Selection (BPS), which we show approaches 
the performance of the optimal solution with reduced complexity.  We additionally 
compare the performance of BPS against two well-known multi-channel, multi-radio 
deployment algorithms across a range of scenarios spanning those typical for 
rural areas to urban areas. In doing so, we achieve up to 160\% of these traffic achieved 
at gateways versus these existing multi-channel, multi-radio algorithms, which are 
agnostic to diverse propagation characteristics across bands.  Moreover, we show 
that, with similar channel resources and bandwidth, the joint use of WiFi and 
white space bands can achieve a served user demand of 170\% that of mesh networks 
with only WiFi bands or white space bands, respectively. Further, through the 
result, we leverage the channel occupancy and spacing impacts on mesh
networks which use white space bands and study the general rules for band selection.

% Future work
%Lastly, we propose to study a multiband application in large 
%scale network deployment using a graph theoretic framework.
%Graph theory is an efficient way to model and solve 
%the deployment problems. The wireless multiband network deployment is formulated
%as a graph theory problem as well as find optional gateway locations.
%In our previous work,
%we focused on solving the channel assignment and access tier mesh nodes deployments
%in multiband scenarios. With the proposed work of graph theoretic solutions to finding gateway nodes
%location, a complete framework for multiband wireless network solution will be 
%built. 
%Also we are going to discuss beamforming in wireless network deployment. 
%Beamforming provides potential gains for reducing the deployment cost and improving
%the fairness in wireless network deployment. We are going to exploit the great
%spacial agility from beamforming for future wireless networks.

