\section{Related Work}
\label{sec:related}

Cognitive Radio could be a powerful tool for the utility of the Spectrum Opportunity~\cite{haykin2005cognitive}.
Analog TV bands will be released for wireless communication brings opportunity to combine current wireless bands and new available bands for performance improvement employing Cognitive Radio methods~\cite{MOAR}. 


%Related research
A bunch of work has been done on radio-scene analysis and channel identification for utility of channel adaptation dating back to Simon Haykin~\cite{haykin2005cognitive}.
Some work of Multi-bands/Multi-channels in
cognitive radios focus on optimize performance, such as avoiding frequency diversity~\cite{rahul2009frequency}. 
In~\cite{OAR} an opportunistic algorithm is introduced to balance the cost of spectrum sensing, channel switching and the gain of these activities.


%Our work
Our work is motivated by prospective releasing band used for TV now and exploit the comparison across all the available bands in the future. 
It is an extension of multi-channel adaptation. 
Most of the published research focus on the stopping rules of spectrum sensing~\cite{sabharwal2007opportunistic, OAR}. In contrast, we use the data and framework to classify the performance across different bands based on the parameters we get from the context information.
%{\bf .} 

