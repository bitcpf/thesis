\begin{abstract} 

Vehicular wireless channels have a high degree of variability, presenting a
challenge for vehicles and infrastructure to remain connected. The emergence of
the white space bands for data usage enables increased flexibility for vehicular
networks with distinct propagation characteristics across frequency bands from
450 MHz to 6 GHz. Since wireless propagation largely depends on the environment
in operation, a historical understanding of the frequency bands' performance in
a given context, could speed multi-band selection as vehicles transition across
diverse scenarios.  In this paper, we leverage knowledge of in-situ operation
across frequency bands with real-time measurements of the activity level to 
select the optimal band for the particular application in use.  To do so, we
perform a number of experiments in typical vehicular topologies.  With a
model based on a Decision Tree and an in-situ training set, we can predict
the throughput on a free channel. 
We can then consider the activity level per band to compute the resulting performance 
one could expect on context information to guide protocol. In the field, we 
exploit the propagation differences experienced per band to show that training on a repeatable 
route can yield vast performance improvements from prior schemes.  We show that minimal  
amounts of training can provide such improvements and that a simple scheme that can allow
multiband adaptation gains when there is insufficient levels of training.


%Probably at this time we could not employ the bus doing the experiments.

%Unused spectrum whitespaces in the currently underutilized analog TV bands are able to exploit for future wireless networks. There are potential room for performance improvement of wireless communication in throughput, power consumption and link fairness extending wireless to these bands. Previous methods are focused on channel adaptation across multiple channels in one band without considering the propogation and other characters among different bands. In this work, we employ the propogation difference for performance prediction of multiband adaptation. To identify the crowded level,we involve an activity level of networks based on the statistics information during a time slot to make the prediction more accuracy. The amount of context information required for multiband adaptation and the influerence of window size for activity level are evaluated in this paper. We conduct indoor and in-field experiments to validate our method. The experimental results demonstrate that our method is able to achieve as ... 


\end{abstract}


