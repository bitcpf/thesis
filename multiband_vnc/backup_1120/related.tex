\section{Related Work}
\label{sec:related}

%Background
%Electromagnetic radio spectrum is a limited natural resource licensed by governments.  Federal Communications Commission (FCC) published a report discussed improving the way to manage this resource in the United States \cite{federal2002spectrum}. 

%Background

Federal Communications Commission (FCC) published a report discussed improving the way to manage the limited natural resource licensed in the United States \cite{federal2002spectrum}. The under utilization of the electromagnetic spectrum of licenced band leads to \emph{Spectrum Opportunity} which provide space for performance improvement based on channel selection \cite{kolodzy2001next}.
\emph{Cognitive Radio} could be a powerful tool for the utility of the \emph{Spectrum Opportunity} \cite{haykin2005cognitive}.
Analog TV bands will be released for wireless communication brings opportunity to combine current wireless bands and new available bands for performance improvement employing \emph{Cognitive Radio} methods \cite{MOAR}. 
%The limited resource requires people to find new ways to share the spectrum improving the efficiency of utility. 
%The under utilization of the electromagnetic spectrum leads to a definition of \emph{Spectrum Opportunity} as a band of frequencies assigned to a primary user, but at a particular time and specific geographic location, the band is not being utilized by that user \cite{kolodzy2001next}.
%Users need a way to find whether they should join the channel without interference other nodes for spectrum sharing.
%In this case, the second user is mentioned who want to join the channel.
%There are two kinds of approaches to employ the under utilization channels. First method is the underlay approach, constraints on the transmission power of second users. 
%Second method is the overlay approach, users need to identify and exploit spatial and temporal spectrum white space \cite{zhao2007survey}.

%Cognitive Radio
%\emph{Cognitive Radio} is an important method for channel selection in spectrum sharing.
%The concept of \emph{Cognitive Radio} is introduced as a novel approach for improving the utilization of the wireless spectrum and the tasks for cognitive radio is summarized in \cite{haykin2005cognitive}. 
%The three on-line cognitive tasks include: \emph{Radio-scene analysis, Channel identification, Transmit-power control and dynamic spectrum management} \cite{haykin2005cognitive}.
%TV channel release

%As analog TV bands will release, wireless communication has opportunity of these bands. Combine different bands to create Multi-bands/Multi-channels system is a new field of \emph{Cognitive Radio} to improve the performance of wireless systems in different
%environments(e.g., as in ~\cite{MOAR}). 

%Related research
A bunch of work has been done on \emph{Radio-scene analysis} and \emph{Channel identification} for utility of channel adaptation dating back to Simon Haykin \cite{haykin2005cognitive}.
Some work of Multi-bands/Multi-channels in
cognitive radios focus on optimize performance, such as avoiding frequency diversity \cite{rahul2009frequency}. 
In \cite{OAR} an opportunistic algorithm is introduced to balance the cost of \emph{spectrum sensing, Channel switching} and the gain of these activities.
%fixme, add more multichannel and add pathloss exponent

%%spectrum sensing
%
%One of the most important components of the congnitive radio concept is the ability to measure, sense, learn and be aware of the parameters related to the radio channel characteristics, availability of spectrum and power, radios operating envrionment \cite{yucek2009survey}. Spectrum sensing becomes the most important component for the estabilishment of congnitive radio. \cite{yucek2009survey}. 

%Adaptation algorithms
%There is a lot of recent research on the design of adaptation algorithms, both rate adaptation and \emph{band/channel} adaptation of cognitive radio systems. These researches are focusing on the \emph{Spectrum sensing and Channel switching strategies}.

%\textbf{Evaluation of Channel Conditions}. Channel condition is the most important component of adaptation. 
%There are two classes of rate adaptation mechanisms that have been developed. 
%These mechanisms are focused on rate adaptation. The first generation adaptation algorithms are loss-triggered. The adaptation algorithm based on the statistics of a previous period of transmission. 
%Second generation rate adaptation schemes diagnose the cause of a loss and appropriately adjust the data rate \cite{biaz2008rate, camp2010modulation}, such as a SNR-triggered protocol. 
%Our work consider both the statistics information of the previous transmission and the dynamic information in context-aware based channel qualification.
%
%\textbf{Evaluation of Adaptation}. Most of the prior work of rate adaptation protocols has investigated the effectiveness via throughput comparison \cite{camp2010modulation}. We also employ throughput as accuracy in the paper to evaluate the performance. 
% 
%\textbf{Primary Second User}. Some other works focus on Multi-channel which bandwidth range limits in 2.4GHz \cite{MOAR} or in a continuous bandwidth considering frequency diversity \cite{rahul2009frequency}. 
%Significant research on the design of channel selection algorithms has been done \cite{radunovic2011dynamic,raniwala2005architecture}. Algorithms are generated for second user to distinguish whether the channel is free or in less utility state as soon as possible \cite{cordeiro2007c}. We are trying to improve the wireless performance taking more frequency bands as a development of these work.
%

%New of our work
Our work is motivated by prospective releasing band used for TV now and exploit the comparison across all the available bands in the future. 
It is an extension of multi-channel adaptation. 
%Our approach classifies the performance based upon combination of in-field measurements and ideal channel conditions on \emph{channel emulator}. 
Most of the published research focus on the stopping rules of spectrum sensing \cite{sabharwal2007opportunistic, OAR}. In contrast, we use the data and framework to classify the performance across different bands based on the parameters we get from the context information.
%{\bf .} 

