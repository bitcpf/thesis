\section{Multiband Adaptation}
\label{sec:model}
In this section, we first focus on the problem formulation for multiband 
adaptation in vehicular wireless links and introduce the context information 
that we use. Based on the analysis of existing problems, we discuss two 
baseline methods for comparison and propose two machine-learning-based 
multiband adaptation algorithms for vehicular channels.

\subsection{Problem Formulation}
The problem we are going to resolve is to find a band has the best throughput in multiple available options as:
\begin{align}
\label{eqn:estimation}
f: (v_{tx}, v_{rx},  P_R^1,..., P_R^n, B^1, ..., B^n,P_N^1,..., P_N^n,) \rightarrow b_{best}
 \end{align}
where $v_{tx}$ and $v_{rx}$ are the velocity of the transmitter and the receiver, 
$P_R^i$ is the received power, $P_N^i$ is the no-802.11 interference signal level and
 %Why we use context information 
$B^i$ is the \emph{busy time}.


To represent the unusable level of the channel, we define \emph{busy time}, $B$,
as the percentage of time when the channel is occupied by 
all competing sources $x_j ( j = 1, 2, 3, ...)$ other than the intended transmitter $y$. 
For 802.11-based transmissions, the busy time on band $i$ could be defined as:
\begin{equation}
\label{eqn:80211activity}
B^i = \frac{\sum_j{\sum_k{\frac{L_k^{x_j}}{R_k^{x_j}}}}}{\sum_k{\frac{L_k^y}{R_k^y}}+\sum_j{\sum_k{\frac{L_k^{x_j}}{R_k^{x_j}}}}+S\sigma}
\end{equation}
where $L$ and $R$ represent the length of packet in bits and the data
rate that packet is transmitted, respectively, for all external sources $x_i$
as compared to the idle slots $S$ times the slot duration $\sigma$ and
the packets from the intended transmitter $y$. When considering the level of
activity of non-802.11 users (e.g., the bands currently licensed to TV 
and other non-802.11 devices), whether the signal level from these
competing sources reaches a level to disrupt communication at the receiver
would define a similar notion of busy time. However, since this depends
on the environment, hardware, coding, and modulation level, we use
the received signal level from non-802.11 interference sources $P_N^i$ 
on band $i$ as an input to our algorithms in various forms as shown 
in~(\ref{eqn:estimation}).

The existing patten embedded in the performance of different bands and 
collected context information e.g. $v_{tx}$, $v_{rx}$, $B_i$, $P_N^i$, $P_R^i$ 
and location information, $g$, could be extracted and help make decisions
for multiband adaptation in a similar context~\cite{meikle2012global} .
%can be used by people tending to have repeatable patterns
%~\cite{meikle2012global} and we could improve the performance with each trip 
%to the region if we could passively observe context along this repeatable 
%pattern. The variables used in each algorithm will be explained for each algorithm 
%later. 

\subsection{Multiband Adaptation Algorithms}
%2 baseline methods: 
%1. band selection based on identifying the the most common class, 
%2. SNR-based band selection
In order to evaluate the proposed multiband adaptation algorithms, 
we construct two simple multiband adaptation methods: (1) We search the
most commonly selected band as the best band in the historical data
and configure the most common band as the final decision. (2) For each band, we build 
a lookup table for throughput $T_{ideal}$ in the ideal channel given $RSSI$ and obtain 
the best band according to following:
\begin{align}
&\max_i T_{ideal}^i\times(1-B^i),
\label{eq:baseline2}
\end{align}
The throughput $T_{ideal}$ is measured with Azimuth ACE-MX channel emulator~\cite{AzimuthACE}. 
The details of system setup and data collection are discussed in Section~\ref{sec:experiment design}. 

Machine learning has been introduced as an important tool in wireless communication
~\cite{haykin2005cognitive}. When the user enters an area, the machine
learning algorithm can learn from the historical data and train a mapping 
function to select the potential optimal band given the input, e.g. RSSI, velocity 
and activity level. We propose two multiband adaptation algorithms based on
two machine learning methods: the k-nearest neighbor (KNN) and the decision tree.

%2 machine learning methods: 
%1. modified KNN, 

{\bf Location based Look-up Algorithm.} KNN is a machine learning method
based on searching closest training data points in the feature space and various
modified versions has been applied successfully for classification~\cite{zhang2006svm}.
In the Location based Look up Algorithm, we search the closest neighbors of 
a testing points by using the parameter one by one in the input set. The 
performance of the found training data points is averaged in each band. Then
the band with the highest throughput performance is selected as the $b_best$.
The Location based look-up algorithm involves the geographic information 
for the band selection comparing with \emph{Machine Learning Algorithm}. 
The process of this algorithm is presented as \ref{algorithms: Location}:
	  \begin{algorithm}
	  \caption{Location based Look-up Algorithm}
	  \label{algorithms: Location}
	  \begin{algorithmic}[1]
	  \REQUIRE  ~~\\
		  $P_R^i \in \{P_R^1,P_R^2 \dots,P_R^n\}$;\\
		  $B^i \in \{B^1,B^2, \dots, B^n\}$;\\
		  $P_N^i \in \{P_N^1,P_N^2 \dots,P_N^n\}$;\\
		  $v_tx$\\
		  $g$: Location Information of Multi-band node\\
		  $Thr_{Area}$: Threshold of a Location\\
		  $Thr_{RSSI}$: Threshold of RSSI\\
		  $Thr_{Velocity}$: Threshold of Velocity\\
		  $Thr_{A Area}$: Threshold of Data Amount for Location\\
		  $Thr_{A RSSI}$: Threshold of Data Amount for RSSI\\
		  $Thr_{A Non 802.11 SI}$: Threshold of Data Amount for Non 802.11 Interference\\
		  $Thr_{A Velocity}$: Threshold of Data Amount for Velocity\\
		  $D^i \in \{D_1,D_2,\dots,D_n\}$: Historical Look up data
\ENSURE ~~\\    
$b_{best}$:Optimal transmission band
\FOR    {$i<=n$}
\STATE Initialize \emph{$Data_{Location},Data_{RSSI},Data_{Velocity}$} to zero matrix;
\WHILE {$Amount(Data_{Location,i})<Thr_{A Area}$}
\STATE $Data_{Location,i} \leftarrow f_{Lookup}(D^i,g,Thr_{Area})$: Find data in $D^i$ whose distance less than $Thr_{Area}$;
\STATE $Thr_{Area}=Thr_{Area} \times 1.1$;
\ENDWHILE

\WHILE {$Amount(Data_{RSSI,i})<Thr_{A RSSI}$}
\STATE $Data_{RSSI,i} \leftarrow f_{Look-up}(D_{Location,i},P_R^i,Thr_{RSSI})$: Find data in $D_{Location}$ the RSSI similar to $P_R^i$ in range $Thr_{RSSI}$;
\STATE $Thr_{RSSI}=Thr_{RSSI} \times 1.1$;
\ENDWHILE

\WHILE {$Amount(Data_{P_N,i})<Thr_{A Non 802.11 SI}$}
\STATE $Data_{RSSI,i} \leftarrow f_{Look-up}(D_{Location,i},P_N^i,Thr_{RSSI})$: Find data in $D_{Location}$ the RSSI similar to $P_N^i$ in range $Thr_{RSSI}$;
\STATE $Thr_{A Non 802.11 SI}=Thr_{A Non 802.11 SI} \times 1.1$;
\ENDWHILE

\WHILE {$Amount(Data_{Velocity,i})<Thr_{A Velocity}$}
\STATE $Data_{Velocity,i} \leftarrow f_{Lookup}(D_{RSSI,i},v_tx,Thr_{Velocity})$: Find data in $D_{RSSI}$ the RSSI similar to $v_tx$ in range $Thr_{RSSI}$;
\STATE $Thr_{Velocity}=Thr_{Velocity} \times 1.1$;
\ENDWHILE \\

\STATE $Tpt_{L,i}=avr(Data_{Velocity,i})$;
\STATE  $Tpt_{e,i}=Tpt_{L,i}\times(1-b_i)$;
\ENDFOR \\  
\STATE $b^*=Max\{Tpt_{e,1},Tpt_{e,1},\dots,Tpt_{e,n}\}$;\\
\end{algorithmic}
\end{algorithm}

For \emph{Location Based Look up Algorithm}, $B_i$ is the \emph{busy time}; $v_i$ 
is the \emph{Velocity} of each band and $r_i$ is the \emph{RSSI} of each band. 
Context information $c$ involves $g$, $v$, and $B^i$. In the 
process of each data point, we have $4$ looking up process to narrow down the data 
points which is similar to the testing data point. First, we find an amount of 
historical data which is near the testing data in a distance range, if the amount of 
historical data which qualify the requirement, we increase the distance range; then we 
narrow the data in the previous looking up qualify the data has similar RSSI in a 
range, if the amount is less than a threshold $Thr_{RSSI}$, the RSSI range will be 
increased; the process is repeated for non 802.11 interference signal and velocity.
At last, the average throughput of the most similar data will be adjust of the 802.11 
\emph{busy time} and tell the best band.

%2. region-based C4.5 (different region sizes)
{\bf Region-based Decision Tree algorithm.} Decision trees are one kind of the most 
widely used machine learning 
algorithms according to its low complexity and stable performance~\cite{banfield2007}.
The decision tree can model the relationship in the training data between the context 
information and the optimal band as a set of tree-like deduction structure. Before 
implementing the training process, we prepare a training set including a group of 
training data points as $\{v_{tx}, v_{rx}, P_R^1, ..., P_R^n,  B^1, ..., B^n, P_N^1, 
..., P_N^n, b_{best}\}$ based on the collected measurements. We obtain $b_best$ by comparing
the throughput performance of all available bands and selecting the band with the highest 
throughput. We choose C4.5 algorithm~\cite{hall2009weka}, an information entropy gain based
and widely used algorithm to build the decision tree used in our system. At each intermediate
node, the learning algorithm calculate the information entropy gain of splitting the rest 
training data points based on each parameter in the input set, e.g. RSSI, velocity or 
activity level. Then it compare and select the parameter with the highest entropy gain
to decide the test condition at each intermediate node until all training data points are
classified. The leaf node indicates the optimal band for prediction in our application
Then the trained decision structure is integrated into the transmission system. With the
collected context information fed into, the decision structure can suggest the potential 
band with the best throughput performance for transmission. 

The relationship between the context information and the best band could be changed at 
different locations because of different propagation environment. To reduce the interference
of training data from different locations, we split the routing of the 
vehicles into several regions and implement the training process based on the historical 
data collected in each region. Then trained decision structure is integrated in our system 
for multiband adaptation in each region. The granularity of region division is one parameter
that affect the training set as well as the performance of the resulting decision tree. We 
evaluate granularity of division in Section~\ref{sec:experiment design}.
