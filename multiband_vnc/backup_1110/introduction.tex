
\section{Introduction}
\label{sec:introduction}


%Background
Drivers can benefit from a wide array of vehicular applications ranging from real-time traffic monitoring and
safety applications to {\it infotainment} applications spanning news, weather, audio, or video streams.  
However, the continuous use of such applications is limited due to the challenge of transmitting over 
highly-dynamic vehicular wireless channels. In such networks, the increasing availability of different 
frequency bands with correspondingly diverse propagation characteristics could allow flexibility and 
robustness of vehicular links. Even with such spectral flexibility, links are extremely tenuous, 
demanding nearly instantaneous decisions in order to remain connected and motivating an algorithm that
can find the appropriate frequency band quickly and according to the current application.

Prior work has considered a number of challenges in
leveraging the digital white space frequencies including spectrum sensing, frequency-agile operation,
geolocation, solving stringent spectral mask requirements, and providing reliable service
in unlicensed and dynamically changing spectrum along with corresponding 
protocols~\cite{shellhammer2009technical}. In particular, there has recently been an acceleration
in spectrum sensing work~\cite{rayanchu2011fluid, kim1996pulse,cabric2004implementation}. Based on 
these works, protocols have been built for multi-channel and/or multi-band wireless operation~\cite{MOAR,
raychaudhuri2003spectrum,sabharwal2007opportunistic}.  Other works have presented a method for searching the most efficient 
transmission channel~\cite{mo2005comparison}, discovering channel information from limited 
measurements~\cite{rayanchu2011fluid, sabharwal2007opportunistic}, and estimating 
channel quality through limit information~\cite{MOAR}. 

While these works have considered spectral activity and developing protocols and algorithms to 
find spectral holes, less of a focus has been on coupling such information with the propagation 
changes that frequency differences of hundreds of MHz to GHz could have on the band decision.  
Moreover, it is well known that propagation greatly depends on the environment in 
operation~\cite{rappaport}.  Thus, 
knowledge of the environment in operation could allow the relationship between received power 
differences across multiple frequency bands to have much greater accuracy.  In this paper, 
we present a multi-band adaptation protocol which leverage in-situ wireless propagation 
measurements and level of per-band activity with machine learning techniques that make a 
band choice according to application-specific performance metrics. To do so, we use an
off-the-shelf platform that allows direct experimentation across four different wireless
frequency bands simultaneously from 450 MHz to 5.8 GHz while maintaining the same physical
and media access layers.

%, it is hard to get a predict SNR just from the pathloss equation. To predict the SNR, a context-aware method is involved in this paper and combine with other methods to predict parameters for multiband adaptation.

%Activity level
%There are other factors will influence the wireless performance in different bands. One factor we are focusing on is the crowd level in different bands. Tons of wireless devices have been used all the world and completed with each other. 
%A concept \emph{activity level} which is the percentage of time the interference nodes occupied is included in this framework to qualify the crowd level. 
%In our framework we analyze the effect of crowd level using \emph{activity level} to estimate the available transmission time for a dominant factor. As a result, we formulate the \emph{Multi-band} adaptation problem to the throughput calculated from known Signal Strength and received packages. We verified the protocol on emulated, indoor and in field experiments to experimentally evaluate our approach. 

%Framwork
%The process of our framework is used to predict the performance involve context-aware information of multi bands. The context-aware information is from ideal channel(Emulator), indoor, in-field experiments. Support Vector Machine(SVM) is employed to connect the dynamic information to the context-aware information.  
%We evaluate our methods by indoor and in field experiments across 4 bands(700MHz, 900MHz, 2.4GHz, 5.8GHz), and show that in certain scenarios our approach can 
%predict the SNR more accuracy than pathloss equation and 
%Through the framework, the performance can be improved by fixme employing multi-band comparing to single wireless band.


%Problem formulation
%To get the two parameters for the performance estimation, both theoretical and context-aware methods are involved in this paper. Based on these parameters and other fixed factors according to specific environment, such as channel type, path-loss exponent and speed of wireless nodes, we are approaching to the basic problem of multi-band adaptation. 
%The basic problem of interest is as follows: \emph{Given the information of one band, get the information of other bands.}
%To solve the problem, 
%we first build a \emph{Ideal Channel Performance Context-Aware database} through the measured data from channel emulator across a wide range of different scenarios. 
%Then, based on the database and the parameters, a tunable threshold connect the \emph{Signal Strength} and the performance of different bands.

%Wireless channels are classed to different types based on the speed, propogation, noise ando so on. A simplistic characterization of channel type example could be a vehicular A in this paper. The basic problem of interest is as follows: \emph{given the channel type and Signal Strength, find the wireless band that achieves the highest throughput.}
%To solve the problem, we first build a \emph{Ideal Channel Look-Up Table} through the measured data from channel emulator across a wide range of different scenarios. Based on this , an ideal throughput for particular signal strength without noise and interference can be connected. The protocol also holds a tunable threshold parameter that determines when the \emph{Signal Strength} is different enough from others.



%Contributions  fixme
The main contributions of our work are as follows:
\begin{itemize}
\item We first formulate the problem of selecting the optimal 
frequency band according to diverse application performance metrics
to be used in the following multiband algorithms.
\item We consider three different algorithms for comparison.  First, we consider a scheme
in which the throughput is achieved on an emulated channel (historical information) for
the current received signal level. We then adjust the predicted best band choice according to the current activity
level (real-time information). 
Second, we consider an approach based on machine learning which
considers prior throughput for a given received signal and activity level
combination.  
Third, we include user location in addition to both the emulated channel and machine learning in addition to the received signal and activity level.
%Third, we consider a second machine learning approach which considers user
%location in addition to the received signal and activity level.
\item We perform V-2-\it{Base Station} experiments to evaluate each algorithm on a repeatable pattern that
%spans multiple environment types (campus, residential, and suburban) with various activity
spans in-field environment with various activity
levels and propagation characteristics within the regions. 
\end{itemize}


%\item We define an \emph{Activity Level} and enroll the concept to dynamic throughput prediction framework. The \emph{Activity Level} can describe the channel interference in time domain.
%\item We propose a simple framework that provides a way to compare the throughput across 4 different bands. Prediction the throughput across multi-band is the precondition to select a band for transmission.  
%\item We build the Look-up Table across different bands, channel models. The LUT refer to the ideal status of the channels across multi bands. It bring benifet for the work in the future.
%We experimentally analyze our framework with simulation and in-field experiments on off-the-shelf hardware platform. The platform is Gateworks 2358 board with 802.11-based radios in different frequency bands. All the radios have a physical layer based on the IEEE 802.11 a/g standard and other sensors for data collection.

The remainder of this paper is organized as follows. In Section II, we present the multiband adaptation problem and proposed algorithms. Section III discusses experimental evaluation of the multiband algorithms. We present related work in Section IV. A summary and discussion of future work is included in Section V.

%Chris: Should we remove the V_2_V experiments in the paper?
