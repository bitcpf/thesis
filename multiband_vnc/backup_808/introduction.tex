
\section{Introduction}
\label{sec:introduction}


%Background
Digital TV is in steps to replace analog TV to free the wireless bandwidth for data usage. These free bands make it possible to extend wireless networks serving in more bands to achive better performance. To exploit the TV white space frequence with existed common wireless communication bands, a number of challenges, including spectrum sensing of both TV signals and wireless devices signals, frequency agile operation, geo-location, stringent spectral
mask requirements, and of course the ability to provide reliable
service in unlicensed and dynamically changing spectrum \cite{shellhammer2009technical} and new protocols to adapt these changes need to be resolved. 

%Related
To put the new free band in utility, the first step is to understand the characters of the bands and the difference among them and find a way to evaluate the performance across different bands.
Over the last few years there has been a rapid increase research done for spectrum sensing in both common communication bands and the White spaces \cite{rayanchu2011fluid, kim1996pulse,cabric2004implementation}. Based on these work, protocols are built for multi-channel, multi-band wireless application \cite{MOAR,raychaudhuri2003spectrum,sabharwal2007opportunistic}.  Such as, in \cite{mo2005comparison}, a method for searching the most efficient transmission channel; learning channel information from limited measurements \cite{rayanchu2011fluid, sabharwal2007opportunistic}; estimate channel quality through limit information\cite{MOAR},etc. However, as far as we know, no work explicitly considered the wireless channel switching across different bands and the resulting performance due to the channel switching.

%Introduce the protocol
In this paper, we present a multi-band adaptation protocol to employ different bands for wireless networks.
The protocol predict the wireless performance across different bands with both theory and context-aware information. And according to these messages, a framework considering the bands' activity is employed to adjust the prediction more accuray for wireless multi-band adaptation. 

%RSSI prediction 
A distinguish character of different bands in an everionment is the signal level related to the pathloss exponent. Also, signal level is a key parameter for wireless network performance evaluation. From pathloss equation, we could calculate the signal level. However, there are a lot of factors will influence the calculation of pathloss equation, the interference, both 802.11 and non-802.11 interference, muti-path loss related to buildings, trees, etc.
Context-aware information is an idea to overcome this problem. For a particular environment, according to the context-aware information, we can predict this parameter more accuracy for the performance evaluate across multi-band.
%, it is hard to get a predict SNR just from the pathloss equation. To predict the SNR, a context-aware method is involved in this paper and combine with other methods to predict parameters for multiband adaptation.


%Activity level
There are other factors will influence the wireless performance in different bands. One factor we are focusing on is the crowd level in different bands. Tons of wireless devices have been used all the world and completed with each other. 
A concept \emph{activity level} which is the percentage of time the interference nodes occupacied is included in this framework to qualify the crowd level. 
In our framework we analyze the effect of crowd level using \emph{activity level} to estimate the available transmission time for a dominant factor. As a result, we formulate the \emph{Multi-band} adaptation problem to the throughput calculated from known Signal Strength and received packages. We verified the protocol on emulated, indoor and in field experiments to experimentally evaluate our approach. 

%Framwork
The process of our framework is used to predict the performance involve context-aware information of multi bands. The context-aware information is from ideal channel(Emulator), indoor, in-field experiments. Support Vector Machine(SVM) is employed to connect the dynamic information to the context-aware information.  
%We evaluate our methods by indoor and in field experiments across 4 bands(700MHz, 900MHz, 2.4GHz, 5.8GHz), and show that in certain scenarios our approach can 
%predict the SNR more accuracy than pathloss equation and 
Through the framework, the performance can be improved by Fixme employing multiband comparing to single wireless band.


%Problem formulation
To get the two parameters for the performance estimation, both theoretical and context-aware methods are involved in this paper. Based on these parameters and other fixed factors according to specific enviroment, such as channel type, pathloss exponent and speed of wireless nodes, we are approaching to the basic problem of multi-band adaptation. 
The basic problem of interest is as follows: \emph{Given the information of one band, get the information of other bands.}
To solve the problem, 
we first build a \emph{Ideal Channel Context-Aware database} through the measured data from channel emulator across a wide range of different scenarios. 
Then, based on the database and the parameters, a tunable threshold connect the \emph{Signal Strength} and the performance of different bands.

%Wireless channels are classed to different types based on the speed, propogation, noise ando so on. A simplistic characterization of channel type example could be a vehicular A in this paper. The basic problem of interest is as follows: \emph{given the channel type and Signal Strength, find the wireless band that achieves the highest throughput.}
%To solve the problem, we first build a \emph{Ideal Channel Look-Up Table} through the measured data from channel emulator across a wide range of different scenarios. Based on this , an ideal throughput for particular signal strength without noise and interference can be connected. The protocol also holds a tunable threshold parameter that determines when the \emph{Signal Strength} is different enough from others.



%Contributions  fixme
The main contributions of our work are as follows:
\begin{itemize}
\item \emph{Propogation} influence across different bands are identified through Contextaware Information.
\item The amount of Context-aware Information for estimation accuracy is evaluated.
\item Evaluate the affect of \emph{Window Size} in prediction.
\item Verify the framework through \emph{simulted, indoor and in-field} experiments and show the improvement.
\end{itemize}


%\item We define an \emph{Activity Level} and enroll the concept to dynamic throughput prediction framework. The \emph{Activity Level} can describe the channel interference in time domain.
%\item We propose a simple framework that provides a way to compare the throughput across 4 different bands. Prediction the throughput across multi-band is the precondition to select a band for transmission.  
%\item We build the Look-up Table across different bands, channel models. The LUT refer to the ideal status of the channels across multi bands. It bring benifet for the work in the future.


We experimentally analyze our framework with simulation and in-field experiments on off-the-shelf hardware platform. The platform is Gateworks 2358 board with 802.11-based Ubiquiti radios(XR2,XR5,and fixme) in differnet bands. All the radios have a physical layer based on the IEEE 802.11 a/g standard, and other sensors for data collection.

The remainder of this paper is organized as follows: In Section II, we discuss related work. Section III presents the background and motivation of the project.  Section IV discusses framework under consideration and provides a brief introduction on the functionality of the framework. Section V introduces the implementation on the WARP platform. Section VI contains experiments used for band switching and evaluation.  A list of future work will be discussed for the project in Section VII.


