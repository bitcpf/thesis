
% Introduce the content of this section
The objective of the mesh network deployment is to minimize the number of 
deployed mesh nodes with the network constraints. In this section we first 
describe the the motivation of frequency agility in mesh network deployment. 
Then we propose a graph-theoretic model for the network deployment problem
with the QoS constraints of an operational wireless mesh network.

% Problem
In our previous work, we address the link communication spectrum adaptation, 
single hop access network deployment, and the multihop access network channel
assignment. Next, we are going to answer the question how to locate the mesh
network access nodes and gateways in multiband spectrum adaptation scenario.
The input of the problem is a target service area given with the parameters 
such as population distribution, spectrum usability, and FCC licensed channels, 
etc. The output is to locate the network infrastructure from the candidates.
Thus, the target area with pre-defined parameters could be modeled as a 
connectivity graph with vertices represented as the centralized traffic demand 
of a certain area and potential access points locations. The edges denote the 
links between the locations. Oppose to previous works,
~\cite{robinson2010deploying,franklin2007node,tang2005interference,irwin2013resource}
we formulate the input connectivity graph as a graph $G = (V,E,F)$, where centralized
traffic demand, access points location candidates and links from a type of access 
point defined by its frequency form a unified connectivity graph. 
We are searching for the output of the problem which is expected to be an graph 
$G' = (V',E',F')$ marks the mesh, gateway nodes and chosen frequency for them. 

The vertices in the modeled input graph represent a set $C$ of  separated target 
area with traffic demands. The set $C$ consists of physical coordinates representing 
target areas where client coverage is desired, analogous to the area to be covered 
in a geometric formulation and the traffic amount need to be served. And also the 
set of potential access points $M$ is a second part of the vertices in the modeled 
graph. The potential locations of access points are assumed known through the 
infrastructure conditions. The vertex set of the input connectivity graph is 
the union of potential access points and centralized traffic demand locations as 
$V = C\cup M$.

The access points types set $F$ is defined by the working frequency bands. It is 
a set of different combinations of frequency bands. The set $E$ in the graph is the 
physical link under protocol model between two vertices according to the access 
point type.

In the output graph, $V'$ includes the chosen access points set $M'$ and served 
traffic demand location set $C'$. The set $F'$ tells the chosen access point 
type of each $M'$. 
%The connectivity and capacity constraints could be defined 
%by the output graph $G'$, as shown in~\ref{eq:graph_coverage} and~\ref{eq:graph_capacity} 

%\begin{equation}
%\label{eq:graph_coverage}
%\frac{\sum{Number\{C'\}}}{\sum{Number\{C\}}} \ge \theta_{coverage} 
%\end{equation}
%$\theta_{coverage}$ is the desired level of coverage for the target area. $C'$ is the 
%served traffic demand location of the target area. 
% 
%\begin{equation}
%\label{eq:graph_capacity}
%C'_n \ge C_n\cdot \theta_{capacity}, C'_n \in C', C_n \in C
%\end{equation}
%$\theta_{capacity}$ is the percentage of satisfied traffic demand for the target area,
%which also include the fairness request in the equation.

The output of the graph could be optimized in several aspects. From the view of network carriers,
the number of access points would be the primary concern $Min{\{Number\{M'\}\}}$.
Through the carriers monthly income from flow charge, maximize the served traffic
demand would be the objective $Max{\{\sum{C'}\}}$. 

